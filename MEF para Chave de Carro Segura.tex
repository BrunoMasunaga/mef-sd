\documentclass[12pt,a4paper]{article}
\usepackage[brazil]{babel}
\usepackage[utf8]{inputenc}

\title{MEF para Chave de Carro Segura}
\author{Bruno Tatsuya Masunaga Santos\\
        Sistemas Digitais 2018.2}
\date{\today}

\begin{document}
\maketitle

\section{Introdução}
Aqui será feita a Apresentação do Projeto e a descrição de um cenário no qual ele se
insere, como uma espécie de justificação genérica da importância do trabalho
realizado. Esta seção deve tentar responder a seguinte pergunta: O que motivou o
trabalho?

\section{Objetivos}
Esta seção poderá ser dividida em Objetivo Geral e Objetivos Específicos, de forma
que o leitor com algum conhecimento básico da área já forme uma ideia dos
procedimentos que foram realizados para desenvolver o Tema. O Objetivo Geral
estabelece o que vai ser feito, ou onde se quer chegar, e os Objetivos Específicos
detalham todas as metas ou etapas necessárias para cumprir o que foi proposto no
Objetivo Geral. Note que as hipóteses de trabalho (sejam elas explícitas ou implícitas)
geralmente delimitam os Objetivos Específicos;

\section{Justificativa}
Nesta seção deverá ser exposto por que o trabalho mereceu ser realizado e em que ele
se diferencia de outros trabalhos correlatos sobre este tema. Esta seção equivale a
uma justificativa específica e deve tentar responder a seguinte pergunta: Qual a
contribuição do trabalho?

\section{Metodologia}
Nesta seção você especifica a forma como o trabalho foi realizado. Que decisão
metodológica foi tomada com relação ao tipo de Entrada de Dados? Arquivo VHDL?
Diagrama Esquemático? Máquina de Estados Finitos? O código-fonte VHDL era uma
Descrição Funcional, Estrutural ou Fluxo de Dados? Qual o tipo de Simulação:
Funcional ou Temporal? O que você espera com o resultado da simulação: resultados
lógicos ou que levam em conta atrasos e tempos de propagação dos sinais? Quais as
características básicas das ferramentas utilizadas? Foi realizada apenas a compilação
do código-fonte VHDL ou também foi feita a Síntese Lógica?

\section{Apresentação dos Dados e Análise dos Resultados}
Respaldado pela Metodologia (que garante que as conclusões obtidas em cima das
simulações sejam válidas), é hora de apresentar o código-fonte VHDL, as simulações e
analisar os resultados.

\section{Apresentação de um Exemplo de Funcionamento do Programa}
Crie uma espécie de mini-tutorial explicando sucintamente como foram feitas as
simulações, e como testar seus programas (GHDL, Modelsim, placas da Altera, etc.)

\section{Conclusão}
Com base na análise dos resultados é possível tirar conclusões que corroborem a
Metodologia de Projeto ou de Simulação. Por isso, na hora de tirar as conclusões,
releia os Objetivos e expresse de forma resumida as análises que demonstram como os
Objetivos foram alcançados. (No caso de os Objetivos não terem sido alcançados, uma
justificativa convincente da impossibilidade ou inviabilidade de se satisfazer as
hipóteses de trabalho pode ser uma Conclusão igualmente válida).

\section{Referência Bibliográfica}
Coloque todas as fontes consultadas, tais como artigos, livros, sites na internet,
informes, relatórios etc.

\end{document}